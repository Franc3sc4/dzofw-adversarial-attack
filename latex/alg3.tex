\subsection{Distributed Stochastic Gradient Free Frank-Wolfe}
This section concerns the discussion of the Distributed Stochastic Grandient Free Frank-Wolfe method implemented in Algorithm 3 for the constraint optimization problem (7) in a distributed setup in which the $M$ workers do not have a central coordinator, instead they exchange information in a peer-to-peer manner. The internode communication network used by the workers is modeled as an undirected simply connected graph $G=(V,E)$, with $V=\{1, \dots, M\}$ the set of nodes and $E$ the set of comunication links. Each node communicates and exchanges information with its own neighbors. Given a node $n$, the set $\Omega_n = \{l \in V | (n,l)\in E\}$ indicates its neighborhood and $d_n = |\Omega_n|$ indicates its degree.\\ The $M \times M$ adjacency matrix $\mathbf{A}=[A_{ij}]$ describes the edges of the graph $G$: $A_{ij}=1$ if $(i,j) \in E$ and $A_{ij}=0$ otherwise. The diagonal matrix $\mathbf{D}=diag(d_1 \dots d_M)$ is used to compute the graph Laplacian $\mathbf{L}=\mathbf{D}-\mathbf{A}$. The normalized Laplacian $\mathbfcal{L} = [\mathcal{L}_{ij}]$ is define to be the matrix
\[
\mathcal{L}_{ij}=
\begin{cases}
	1 & \text{if $i=j$ and }d_i\ne0, \\
	-\frac{1}{\sqrt{d_id_j}} & \text{if $i$ and $j$ are adjacent,}\\
	0 & \text{otherwise.}
	
\end{cases}
\]
We can write $\mathbfcal{L} = \mathbf{D}^{-1/2}\mathbf{L}\mathbf{D}^{-1/2}$. The normalized Laplacian $\mathbfcal{L}$ is used to compute the weghted matrix $\mathbf{W} = \mathbb{1}- \mathbfcal{L}$.\\
Unlike the previous algorithms, the perturbation \mbox{\boldmath$ \delta$}$_0$ is initialized as a null $M \times d$ matrix, rather than a vector: this matrix is given in input to each worker, which only considers the rows corresponding to the perturbations computed by its neighbors. This trick simplifies the distributed architecture, without changing the functioning of the algorithm.\\
At every iteration, each worker $i$ exchanges its current iterate with its neighbors and averages the received iterates. Then, each worker $i$ applies the I-RDSA scheme to compute its local gradient estimation $\mathbf{g}_t^i$ and exchanges with the neighbors the vector $\mathbf{G}_t^i$, which is an averaged version of the gradient. Finally, when all the $\mathbf{G}_t^i$ have been calculated, they are involved in a weighted sum computed by each worker.\\
For what said until now, the implementation requires two round of communication: one for the iterates $\bar{\mathbf{x}}_t^i$ and one for the gradients $\bar{\mathbf{g}}_t^i$. The performance depends on how well these vectors are tracked across the network.\\
At the end of the t-th iteration, each worker computes the Frank-Wolfe update of the iterate \mbox{\boldmath$ \delta$}$_{t+1}^i$ using (5) on $\bar{\mathbf{g}}_t^i$ and (6) on $\bar{\mathbf{x}}_t^i$ with $\gamma_t = t^{-1/2}$.\\
\begin{algorithm}
	\caption{Distributed SGF FW}\label{distributed}
	\textbf{Input} Batches of images $\{\mathbf{x}_i\}_{i=1}^M$, batches of labels $\{\mathbf{y}_i\}_{i=1}^M$, loss function $F$, number of queries $T$, number of workers $M$, image dimension $d$, tolerance $\varepsilon$, number of sampled directions $m$, adjacency matrix $\mathbf{A}$, weight matrix $\mathbf{W}$.\\
	\textbf{Output} $\bar{\mbox{\boldmath$ \delta$}}_T^i\;\;\forall i \in \{1\dots M\}$
	\begin{algorithmic}[1]		
		\State For $i=1,\dots, M$ initialize \mbox{\boldmath$ \delta$}$_0^i =\mathbf{0} $ and $\bar{\mathbf{g}}_0^i = \mathbf{0}$
		\For {$t = 1, \dots, T$}
		\State Compute the parameter required for the computation of the I-RDSA schemes: 
		${\scriptsize c_t =\frac{2\sqrt{m}}{d^{3/2}(t+8)^{1/3}}}$
		\State In the first round of communication, each worker $i$ approximates the average iterate: \newline
		\[\bar{\mbox{\boldmath$ \delta$}}_t^i \leftarrow \sum_{j=1}^{M} W_{ij}\mbox{\boldmath$ \delta$}_t^i\]
		\State Each worker $i$ computes the I-RDSA scheme:\newline 
		Sample $\{\mathbf{z}_{j,t}\}_{j=1}^m \sim\mathcal{N}(0,\mathbb{1}_d)$ \newline
		\[\mathbf{g}_t^i = \frac{1}{m} \sum_{j=1}^{m} \frac{F(\mathbf{x}_i + \mbox{\boldmath$ \delta$}_t + c_t\mathbf{z}_{j,t}, \mathbf{y}_i) - F(\mathbf{x}_i + \mbox{\scriptsize\boldmath$ \delta$}_t, \mathbf{y}_i)}{c_t}\mathbf{z}_{j,t}\]
		
		\State  Each worker $i$ computes:
		\[ \mathbf{G}_t^i = \bar{\mathbf{g}}_{t-1}^i + \mathbf{g}_t^i - \mathbf{g}_{t-1}^i \]
		\State In the second round of communication, each worker $i$ approximates the average gradient:
		\[ \bar{\mathbf{g}}_t^i \leftarrow \sum_{j=1}^{M} W_{ij}\mathbf{G}_t^j  \]
		
		\State Each worker $i$ computes $\mathbf{v}_t^i = - \varepsilon sign(\bar{\mathbf{g}}^i_t)$.
		\State Each worker $i$ updates:
		\[\mbox{\boldmath$ \delta$}_{t+1}^i \leftarrow (1-\gamma_t)\bar{\mbox{\boldmath$ \delta$}}_t^i + \gamma_t\mathbf{v}_t^i\]
		\EndFor
		
	\end{algorithmic}
\end{algorithm}
 using $c_t = \frac{2\sqrt{m}}{d^{3/2}(t+8)^{1/3}}$