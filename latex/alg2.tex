\subsection{Decentralized Variance-Reduced Stochastic Gradient Free Frank-Wolfe}
In this section the goal is to solve another special case of (1), that is
\begin{equation}
	\mbox{\boldmath$ \delta$}' = \argmin_{\lVert \mbox{\boldmath$\scriptstyle \delta$}\rVert_{\infty}\leq\varepsilon} \sum_{i=1}^M \sum_{j=1}^n \mathbb{E}_{(\mathbf{x},y)\sim \mathcal{P}_{i,j}}[- F_{i,j}(\mathbf{x}+\mbox{\boldmath$ \delta$},y)].
\end{equation}
The implementation takes into account the SPIDER variance reduction technique, which is built for dynamic tracking, while avoiding excessive querying to oracles and ultimately reducing query complexity.\\
\indent In the Decentralized Variance-Reduced Stochastic Gradient Free Frank-Wolfe method implemented in Algorithm \ref{variance-reduced},
at the beginning of each period, that is when $mod(t,q)=0$ with period parameter $q \in \mathbb{N}_{+}$, each worker employs the KWSA scheme for the computation of its gradient estimation. In all the other cases, each worker selects a mini-batch of local component functions $\{F_{i,j}\}_{j\in S_2}$ and uses the RDSA scheme to estimate and update its gradient. Then, the master node computes the average of the received estimated gradients and calculates the new perturbation using the equation (6) with $\gamma_t = \frac{2}{t+8}$.\\
\indent For what concerns the data, each worker receives the same amount $S_1$ of images (selected to be balanced with respect to the ten classes) and the corresponding labels from the dataset. In particular, the same image is not given to different workers. For both the KWSA and the RDSA schemes, each worker samples a batch of $\frac{S_1}{M}$ images, from the received ones. For the KWSA scheme, this sampling is done $d$ times, one for each component of the gradient.  
\begin{algorithm}
	\caption{Decentralized Variance-Reduced SGF FW}\label{variance-reduced}
	\textbf{Input} Batches of images $\{\mathbf{x}_i\}_{i=1}^M$, batches of labels $\{\mathbf{y}_i\}_{i=1}^M$, loss function $F$, number of queries $T$, number of workers $M$, image dimension $d$, tolerance $\varepsilon$, number of sampled images and labels $S_1$, number of sampled component functions $S_2$, total number of component functions $n$, period $q$.\\
	\textbf{Output} Universal perturbation's history.
	\begin{algorithmic}[1]		
		\State Initialize \mbox{\boldmath$ \delta$}$_{0} = \mathbf{0}$.
		\For {$t = 0, \dots, T-1$}
		\State Master node computes parameters required for the computation of the KWSA and RDSA schemes: 
		{\scriptsize\[\eta_1 =\frac{2}{d^{1/2}(t+8)^{1/3}},\;\;\;\eta_2 =\frac{2}{d^{3/2}(t+8)^{1/3}}\]}
		\State Each worker $i$ computes:
		\If {$mod(t,q)=0$}
		\State For each component $k$ of the gradient, draw $\frac{S_1}{M}$ images $\mathbf{x}'_i$ and the corresponding labels $\mathbf{y}'_i$ from the batches $\mathbf{x}_i$ and $\mathbf{y}_i$, for a total of $\frac{S_1}{M}d$ samples. For $k=1\dots d$ compute \newline
		\[\small \mathbf{e}_k^T\mathbf{g}_i = \frac{1}{n}\sum_{j=1}^{n}\frac{F_{i,j}(\mathbf{x}'_i + \mbox{\boldmath$ \delta$}_t + \eta_1\mathbf{e}_k, \mathbf{y}'_i) - F_{i,j}(\mathbf{x}'_i+\mbox{\boldmath$ \delta$}_t, \mathbf{y}'_i) }{\eta_1} \]
		
		\State Update $\mathbf{g}_{i,t} = \mathbf{g}_i$.
		\Else
		\State Sample $\mathbf{z} \sim\mathcal{N}(0,\mathbb{1}_d)$. Draw $S_2$ component functions and $\frac{S_1}{M}$ images $\mathbf{x}'_i$ and the corresponding labels $\mathbf{y}'_i$ from the batches $\mathbf{x}_i$ and $\mathbf{y}_i$. Then compute
		\[\small \mathbf{g}_i = \frac{1}{|S_2|}\sum_{j\in S_2}\frac{F_{i,j}(\mathbf{x}'_i + \mbox{\boldmath$ \delta$}_t + \eta_2\mathbf{z}, \mathbf{y}'_i) - F_{i,j}(\mathbf{x}'_i+\mbox{\boldmath$ \delta$}_t, \mathbf{y}'_i) }{\eta_2}\mathbf{z} \]
		\[\small-\frac{F_{i,j}(\mathbf{x}'_i + \mbox{\boldmath$ \delta$}_{t-1} + \eta_2\mathbf{z}, \mathbf{y}'_i) - F_{i,j}(\mathbf{x}'_i+\mbox{\boldmath$ \delta$}_{t-1}, \mathbf{y}'_i) }{\eta_2}\mathbf{z} \]
		\State Update $\mathbf{g}_{i,t} = \mathbf{g}_i +\mathbf{g}_{i,t-1} $.
		\EndIf
		\State Each worker $i$ pushes $\mathbf{g}_{i,t}$ to the master node.
		\State Master node computes the gradients average:
		\[\mathbf{g}_t = \frac{1}{M} \sum_{i=1}^{M} \mathbf{g}_{i,t}.\]
		\State Master node computes $\mathbf{v}_t = - \varepsilon sign(\mathbf{g}_t)$.
		\State Master node computes \mbox{\boldmath$ \delta$}$_{t+1} = (1-\gamma_t)\mbox{\boldmath$ \delta$}_t + \gamma_t\mathbf{v}_t$ and sends it to all nodes.
		\EndFor
	\end{algorithmic}
\end{algorithm}
