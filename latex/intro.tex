\section{Introduction}
\subsection{Adversarial Attacks and Universal Adversarial Perturbations}
TODO: What are adversarial attacks?

Universal perturbations are small perturbations that, when applied to the input images, are able
to fool a state-of-the-art deep neural network classifier. These perturbations are quasi-imperceptible for humans,
due to their small norm, and therefore they are difficult to detect. However, what makes universal perturbations special,
is their capability to generalize well both on a large dataset and across different deep neural network architectures.
In fact, an important property of universal perturbations is that they are image-agnostic. This means that
they don't depend on a single image, but rather they are able to cause label estimation change for most of
the images contained in a dataset.

It is important to notice that there's no unique universal perturbation: different random shuffling of the data used
to compute the perturbation can lead to a diverse set of universal perturbations.

Furthermore, universal perturbations mostly make natural images classified with specific labels, called dominant labels.
These dominant labels are not determined a priori, but rather they are automatically found by the algorithm that
computes the universal perturbation. In paper \textcolor{red}{XXX}, the authors hypothesize that "these dominant labels occupy large
regions in the image space, and therefore represent good candidate labels for fooling most natural images".

Finally, although some fine-tuning strategies can be adopted to make deep neural networks more robust to adversarial
attacks, they are not sufficient to make the classifier immune to the attacks. In fact, the fine-tuning of a classifier with universal perturbations leads to an improvement in the classification of
perturbed images, however it is always possible to find a small universal perturbation that can fool the network.

% non-convex problem

% the objective is not to find the smallest perturbation that fools most of the data points,
% but rather to find one such perturbation with sufficiently small norm

% universal perturbations exploit some geometric correlations between different parts in the decision
% boundary of the classifier

