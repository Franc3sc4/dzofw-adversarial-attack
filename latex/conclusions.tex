\section{Conclusions}
In this report we have focused on the problem of producing universal adversarial perturbations by analizing three
Stochastic Gradient Free Frank-Wolfe algorithms. In particular, we have shown that the perturbations created by
Decentralized and Distributed SGF FW algorithms follow a similar and more clear pattern compared to the Decentralized
Variance-Reduced SGF FW algorithm. In particular, we can clearly see that the reproduced pattern has a 3 shape, which
leads the majority of handwritten digits to be misclassified as 3. This can be explained by the concept of \textit{dominant labels},
mentioned in Section \ref{section:perturb}. In fact, number 3 is a wide number, that covers most of the space in the image. Therefore, a
perturbation with a 3 shape can easily lead to the misclassification of smaller numbers such as 1 and 7, which occupy
less space in the image. On the contrary, the perturbations produced by the Decentralized Variance-Reduced SGF FW algorithm,
don't have a clear pattern and the noise associated with them looks randomly spread.

Furthermore, the algorithm that reached better results in terms of misclassification is Algorithm \ref{decentralized},
which lowered the classifier's accuracy to 55\%. In this sense, the worst algorithm was \ref{variance-reduced} since
it was unable to lower the classifier's accuracy below 84\%.


% confronto tra i nostri metodi:
% - confronto pattern --> how the noise is spread in the perturbation
% - confronto accuracy --> small accuracy, best algorithm
% - confronto running-time?