\section{Conclusions}
In this report we focused on the problem of producing Universal Adversarial Perturbations by analyzing three
Stochastic Gradient Free Frank-Wolfe algorithms.

First of all, we have shown that the perturbations created by Decentralized (\ref{decentralized}) and Distributed (\ref{distributed})
SGF FW algorithms present a similar and more clear pattern compared to the Decentralized Variance-Reduced SGF FW
algorithm (\ref{variance-reduced}). In particular, we can clearly see that the reproduced pattern has a 3-shape, which
leads the majority of handwritten digits to be misclassified as 3. This can be explained by the concept of \textit{dominant labels},
mentioned in Section \ref{section:perturb}. In fact, number 3 is a wide number, that covers most of the space in the image. Therefore, a
perturbation with a 3-shape can easily lead to the misclassification of smaller numbers such as 1 and 7, which occupy
less space in the image. On the contrary, the perturbations produced by the Decentralized Variance-Reduced SGF FW algorithm,
don't have a clear pattern and the noise associated with them looks randomly spread.

Secondly, the algorithm that reached better results in terms of misclassification is Algorithm \ref{decentralized},
which lowered the classifier's accuracy to 55\%. In this sense, the worst algorithm was \ref{variance-reduced} since
it was unable to decrease the classifier's accuracy below 84\%. A possible improvement could be a better tuning of the hyperparameters; in fact in our experiments we considered quite low values of the number $M$ of workers, the number $T$ of queries and the number $S_2$ of sampled component functions and we also set the period parameter $q$ so that the KWSA scheme was called at most 4 times. 

If we compare the execution time of the three algorithms, we can observe that algorithms \ref{decentralized} and \ref{distributed}
are much faster then algorithm \ref{variance-reduced}. This is due to the fact that Algorithm \ref{variance-reduced}
employs KWSA, which is very expensive in terms of CPU time.

Compared to the experiments described in paper \cite{A3}, we obtained slightly higher error rates with algorithm
\ref{decentralized}, while, with algorithm \ref{distributed} , we achieved lower error rates. The latter result can be
explained by the fact that we chose to use the I-RDSA scheme with $m=15$ instead of the KWSA scheme, to reduce the time
complexity of the algorithm, although the KWSA scheme gives a more precise gradient approximation. 

Furthermore, the
distributed setting of Algorithm \ref{distributed} naturally leads to a less precise gradient approximation than the one of Algorithm \ref{decentralized}, due to the
fact that each node has access only to the computations made by its neighbors. Therefore, the choice of a less precise
method to compute the gradient, i.e. the I-RDSA scheme with a small value for $m$, makes the resulting perturbation
less performing. Nevertheless, the attack performed with the perturbation obtained from Algorithm \ref{distributed} is satisfying enough, since random noise resulted to be much less effective.

Finally, in our last experiment we proved that the perturbation created with algorithm \ref{decentralized} on LeNet-5's loss function is universal
not only with respect to the MNIST dataset, but also across different deep neural network architectures, such as AlexNet.
% confronto tra i nostri metodi:
% - confronto pattern --> how the noise is spread in the perturbation
% - confronto accuracy --> small accuracy, best algorithm
% - confronto running-time?

% confronto con i risultati del paper: